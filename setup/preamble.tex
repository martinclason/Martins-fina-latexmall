% Det här är den jobbigaste filen. Oftast behöver man inte in här och rota men det är här som alla paket och bibliotek importeras.
% Det är även här man kan ändra sidhuvuden och sådant.

% Utgått mycket från Aalborg university report
\documentclass[11pt,a4paper]{article}
\usepackage{luatextra}
%%%%%%%%%%%%%%%%%%%%%%%%%%%%%%%%%%%%%%%%%%%%%%%%
% Language, Encoding and Fonts
% http://en.wikibooks.org/wiki/LaTeX/Internationalization
%%%%%%%%%%%%%%%%%%%%%%%%%%%%%%%%%%%%%%%%%%%%%%%%
% Make latex understand and use the typographic
% rules of the language used in the document.
\usepackage[swedish,english]{babel}
%\usepackage{polyglossia}

% Fonts
\setmainfont{Times New Roman}
\setsansfont{Arial}
%\usepackage{bold-extra} % Vet inte vad denna gör
%%%%%%%%%%%%%%%%%%%%%%%%%%%%%%%%%%%%%%%%%%%%%%%%
% Graphics and Tables
% http://en.wikibooks.org/wiki/LaTeX/Importing_Graphics
% http://en.wikibooks.org/wiki/LaTeX/Tables
% http://en.wikibooks.org/wiki/LaTeX/Colors
%%%%%%%%%%%%%%%%%%%%%%%%%%%%%%%%%%%%%%%%%%%%%%%%
% load a colour package
\usepackage{xcolor}
\definecolor{aaublue}{RGB}{33,26,82}% dark blue
% The standard graphics inclusion package
\usepackage{graphicx}
% Set up how figure and table captions are displayed
\usepackage{caption}
\captionsetup{%
  font=footnotesize,% set font size to footnotesize
  labelfont=bf % bold label (e.g., Figure 3.2) font
}
% Make the standard latex tables look so much better
\usepackage{array,booktabs}
% Enable the use of frames around, e.g., theorems
% The framed package is used in the example environment
\usepackage{framed}
\usepackage{float} % position with H

%%%%%%%%%%%%%%%%%%%%%%%%%%%%%%%%%%%%%%%%%%%%%%%%
% Mathematics
% http://en.wikibooks.org/wiki/LaTeX/Mathematics
%%%%%%%%%%%%%%%%%%%%%%%%%%%%%%%%%%%%%%%%%%%%%%%%
% Defines new environments such as equation,
% align and split 
\usepackage{amsmath}
% Adds new math symbols
\usepackage{amssymb}
% Use theorems in your document
% The ntheorem package is also used for the example environment
% When using thmmarks, amsmath must be an option as well. Otherwise \eqref doesn't work anymore.
\usepackage[framed,amsmath,thmmarks]{ntheorem}
\numberwithin{equation}{section}% To get section number in equation numbering
%\usepackage{fourier} % Parallel slant
\newcommand{\parallelsum}{\mathbin{\!/\mkern-5mu/\!}}
\usepackage{IEEEtrantools}
\usepackage{cases}


%%%%%%%%%%%%%%%%%%%%%%%%%%%%%%%%%%%%%%%%%%%%%%%%
% Page Layout
% http://en.wikibooks.org/wiki/LaTeX/Page_Layout
%%%%%%%%%%%%%%%%%%%%%%%%%%%%%%%%%%%%%%%%%%%%%%%%
%\usepackage{showframe} % Shows documents layout and margin boxes.
%\usepackage{layout}
%\usepackage{titleps} % Believe change in newgeometry after \begin{document}  requires this package, laoded with \usepackage[pagestyles]{titlesec} instead
% Change margins, papersize, etc of the document
\usepackage[
  left=28mm,
  right=28mm,
  bottom=15mm,
  top=15mm,
  includehead,
  includefoot
  ]{geometry}
% Modify how \chapter, \section, etc. look
% The titlesec package is very configureable
%\usepackage{titlesec}
%\titleformat{\chapter}[display]{\normalfont\huge\bfseries}{\chaptertitlename\ \thechapter}{20pt}{\Huge}
%\titleformat*{\section}{\normalfont\Large\bfseries}
%\titleformat*{\subsection}{\normalfont\large\bfseries}
%\titleformat*{\subsubsection}{\normalfont\normalsize\bfseries}
%\titleformat*{\paragraph}{\normalfont\normalsize\bfseries}
%\titleformat*{\subparagraph}{\normalfont\normalsize\bfseries}

% Font management
\usepackage{fontspec}
\usepackage[pagestyles]{titlesec}
\usepackage{geometry}
\defaultfontfeatures{Ligatures=TeX}
%\usepackage{libertine}

% Headings
\titlespacing\section{0pt}{12pt plus 4pt minus 2pt}{0pt plus 2pt minus 2pt}
\titlespacing\subsection{0pt}{12pt plus 4pt minus 2pt}{0pt plus 2pt minus 2pt}
\titlespacing\subsubsection{0pt}{12pt plus 4pt minus 2pt}{0pt plus 2pt minus 2pt}

%\newcommand\liningnumbers{\addfontfeature{Numbers=Lining}}
%\renewcommand{\linenumberfont}{\scriptsize\addfontfeatures{Numbers={Lining, Monospaced}}}
%\addfontfeature{Numbers=Lining}

\titleformat*{\section}{\fontsize{16}{18}\bfseries\sffamily}
\titleformat*{\subsection}{\fontsize{14}{16}\bfseries\sffamily}
\titleformat*{\subsubsection}{\fontsize{12}{14}\bfseries\sffamily}

% Paragraph formatting
\usepackage[parfill]{parskip}

% Bullet styling
\renewcommand{\labelitemi}{$-$}
\renewcommand{\labelitemii}{$\diamond$}
\renewcommand{\labelitemiii}{$\circ$}

% Clear empty pages between chapters
\let\origdoublepage\cleardoublepage
\newcommand{\clearemptydoublepage}{%
  \clearpage
  {\pagestyle{empty}\origdoublepage}%
}
\let\cleardoublepage\clearemptydoublepage

% Do not stretch the content of a page. Instead,
% insert white space at the bottom of the page
\raggedbottom
% Enable arithmetics with length. Useful when
% typesetting the layout.
\usepackage{calc}

% Headers and footers

\newpagestyle{titlestyle}{
	%\setheadrule{1pt}
    %\sethead{aaa}{aaa}{aaa}
    \sethead{{\begin{tabular}[b]{l@{}l}
                Inlämningsuppgift\\ 
                Elektronik och mätteknik\\
                TSTE05\\
             \end{tabular}}}
            {}
            {{\begin{tabular}[b]{r@{}l}
                Linköpings universitet - ISY\\ 
				Teknisk fysik och elektroteknik\\
				Höstterminen 2017\\
             \end{tabular}}}
    \addtolength\headheight{40pt}
}

\newpagestyle{main}{
	\setheadrule{0.55pt}
    \setfootrule{0.55pt}
    \sethead{Inlämningsuppgift 5-000}
            {}
            {\thesection . \sectiontitle}
    \setfoot{TSTE05}
            {\thepage}
            {Martin Clason}
}


%%%%%%%%%%%%%%%%%%%%%%%%%%%%%%%%%%%%%%%%%%%%%%%%
% Bibliography
% http://en.wikibooks.org/wiki/LaTeX/Bibliography_Management
%%%%%%%%%%%%%%%%%%%%%%%%%%%%%%%%%%%%%%%%%%%%%%%%
\usepackage[backend=bibtex,
  bibencoding=utf8
  ]{biblatex}
\addbibresource{bib/mybib}

%%%%%%%%%%%%%%%%%%%%%%%%%%%%%%%%%%%%%%%%%%%%%%%%
% Misc
%%%%%%%%%%%%%%%%%%%%%%%%%%%%%%%%%%%%%%%%%%%%%%%%
% Add bibliography and index to the table of
% contents
\usepackage[nottoc]{tocbibind}
% Add the command \pageref{LastPage} which refers to the
% page number of the last page
\usepackage{lastpage}
% Add todo notes in the margin of the document
\usepackage[
%  disable, %turn off todonotes
  colorinlistoftodos, %enable a coloured square in the list of todos
  textwidth=\marginparwidth, %set the width of the todonotes
  textsize=scriptsize, %size of the text in the todonotes
  ]{todonotes}

%%%%%%%%%%%%%%%%%%%%%%%%%%%%%%%%%%%%%%%%%%%%%%%%
% Hyperlinks
% http://en.wikibooks.org/wiki/LaTeX/Hyperlinks
%%%%%%%%%%%%%%%%%%%%%%%%%%%%%%%%%%%%%%%%%%%%%%%%
% Enable hyperlinks and insert info into the pdf
% file. Hypperref should be loaded as one of the 
% last packages
\usepackage{hyperref}
\hypersetup{%
	pdfpagelabels=true,%
	plainpages=false,%
	pdfauthor={Author(s)},%
	pdftitle={Title},%
	pdfsubject={Subject},%
	bookmarksnumbered=true,%
	colorlinks=false,%
	citecolor=black,%
	filecolor=black,%
	linkcolor=black,% you should probably change this to black before printing
	urlcolor=black,%
	pdfstartview=FitH%
}

% Other
\usepackage{blindtext}
\usepackage{lipsum} % To write out random latin text.